     % !TeX encoding = UTF-8
\documentclass[a4paper,11pt]{article}
\usepackage{amssymb}
\usepackage{amsmath}
\usepackage{IEEEtrantools}
\usepackage[normalem]{ulem}
\newcommand{\project}[1]{\textbf{#1}\xspace}
\newcommand{\SECURITY}{\project{Elysian 2}}
\newcommand{\TheProject}{\SECURITY}

%\ifdefined\final
%\else
%\newcommand{\final}{}
%\fi

\def \final {}

%% Included Packages
%\usepackage{a4wide}
\usepackage[cm]{fullpage}
\usepackage{fancyhdr}
\usepackage{txfonts}
\usepackage{path}
\usepackage{url}
%\usepackage{multicol}
\usepackage{graphicx}
\usepackage{lscape} % for landscape
\usepackage[official,right]{eurosym}
\usepackage{numbering}
\usepackage{array}
\usepackage{paralist} % for compactitem/compactenum/... environments
\usepackage{xspace} % to allow automatic spacing of the project acronym but still observing punctuation.
% \usepackage{hyperref}
\usepackage{natbib} % squeeze biblio entries
\setlength{\bibsep}{0.0pt}
\usepackage{titlesec}
\usepackage{wrapfig} % Wrap text around figures
\usepackage{microtype} % Fix underfull/overfull text boxes

%% For file splitting
\usepackage{pageslts} %  theCurrentPage
\usepackage{calc} % theCurrentPage+1
\usepackage{blindtext} % Lorem ipsum dolor ...

\newcounter{FirstPage} % counter, for first page number
\setcounter{FirstPage}{1} % Set FirstPage to page 1

\newwrite\BatchFile% Batch-Datei
\immediate\openout\BatchFile=split.sh

\newcommand{\Split}[1]{%
\immediate\write\BatchFile{pdftk\space \jobname.pdf\space cat \theFirstPage -\theCurrentPage\space output\space \jobname-#1.pdf dont_ask}% 
\setcounter{FirstPage}{\theCurrentPage+1}% set FirstPage to new first page
}

% For for-loops
\usepackage{pgffor}
\usepackage{forloop}

\usepackage{longtable}

%% Format workpackage descriptions in a standard way
\usepackage{Workpackage}


%% Use the standard commenting package: change this to final in the submitted version to remove comments
\ifdefined\final
\usepackage[final]{comments}
\else
\usepackage[draft]{comments}
\fi

%\usepackage{ifthen}
\usepackage{fp}

\newcommand{\hl}[1]{\fcolorbox{red}{red}{#1}}

% Euro symbol for financial information
\newcommand{\euros}{\euro{}}

%% Standard page formatting commands
\renewcommand{\thesection}{\arabic{section}}

% Set the page margins etc.  Superceded.
% \advance\topmargin by -1.3cm
% \advance\textheight by 1.3cm
% \advance\leftmargin by -12mm
% \advance\textwidth by 12mm
% \advance\columnsep by 4mm
\parindent= 0pt
\parskip=0.125\medskipamount
\raggedbottom
\pagestyle{empty}

% Fix awkward figures
\renewcommand{\textfraction}{0.1}
\renewcommand{\topfraction}{1.0}
\renewcommand{\floatpagefraction}{0.9}

% Include images from the images directory
\newcommand\includeimage[2][]{%
\IfFileExists{images/#2.pdf}{%
\includegraphics[#1]{images/#2.pdf}%
}
{
\IfFileExists{images/#2}{%
\includegraphics[#1]{images/#2}%
}
{%
\vspace{3in}%
}
}
}


\newcommand{\mydate}{\today}
\setcounter{page}{0}

%% Control table of contents - depth 1 = section, 2 = subsection
\addtocontents{toc}{\protect\setcounter{tocdepth}{2}}

%% Header and Footer
\cfoot{\thepage}
\lfoot{\TheProject}
\rfoot{\mydate}

\lhead{}
\chead{}
\rhead{}
\renewcommand{\headrulewidth}{0pt}
\renewcommand{\footrulewidth}{0.4pt}% default is 0pt

\pagestyle{fancy}

% Squeeze wasted space
\titlespacing{\paragraph}{%
  0pt}{%              left margin
  0.5\baselineskip  plus 0.1\baselineskip minus   0.1\baselineskip}{% space before (vertical)
  1em}%               space after (horizontal)

\titlespacing{\subsubsection}{%
  0pt}{%              left margin
  0.5\baselineskip  plus 0.1\baselineskip minus   0.1\baselineskip}{% space before (vertical)
  1em}%               space after (horizontal)

\titlespacing{\subsection}{%
  0pt}{%              left margin
  0.5\baselineskip  plus 0.1\baselineskip minus   0.1\baselineskip}{% space before (vertical)
  1em}%               space after (horizontal)

\titlespacing{\section}{%
  0pt}{%              left margin
  0.5\baselineskip  plus 0.1\baselineskip minus   0.1\baselineskip}{% space before (vertical)
  1em}%               space after (horizontal)
\sloppy

\renewcommand{\baselinestretch}{0.91}

\usepackage{mdframed}
\usepackage[dvipsnames]{xcolor}
\newenvironment{commentary}[1]{%
\begin{mdframed}[backgroundcolor=blue!15]
% \noindent\fcolorbox{black}{boxgray}{%
%    \parbox{\textwidth}{%
\textbf{#1}%
}%
{%
\end{mdframed}
\bigskip\bigskip
}

\newcommand{\horizontallinedef}[1]{{\vspace{6pt}\color{ProcessBlue}\hrule height#1pt\vspace{6pt}}}

%\newcommand{\horizontalline}{{\vspace{6pt}\color{ProcessBlue}\hrule height1pt\vspace{6pt}}}
\newcommand{\horizontalline}{\horizontallinedef{1}}

\newcommand{\boxpara}[1]{{#1}\newline{}\hspace*{\fill}\rule[0.4ex]{0.9\linewidth}{1pt}\hspace{\fill}{}}

\hbadness=10000
\hfuzz=8pt

\usepackage{textcomp}  % Copyright symbol

% Partners
\newparticipant{UOD}{University of Dundee}{UOD}{UK}{UOD}
\newparticipant{SA}{University of St Andrews}{USTAN}{UK}{USTAN}
\newparticipant{IBM}{IBM Israel Science \& Technology Ltd.}{IBM}{IL}{IBM}
\newparticipant{SOPRA}{Sopra-Steria Ltd.}{SOPRA}{UK}{SOPRA}
\newparticipant{SCCH}{Software Competence Center Hagenberg}{SCCH}{AT}{SCCH}
\newparticipant{YAG}{Yagaan Software Security}{YAG}{FR}{YAG}

\newcommand{\USTAN}{\SA}
\newcommand{\USTANnr}{\SAnr}
\newcommand{\USTANshort}{\SAshort}
\newcommand{\USTANshorter}{\SAshorter}
\newcommand{\USTANlong}{\SAlong}


% Coordinator
\newcommand{\coordnr}{1}
\newcommand{\coord}{\participant{\coordnr}}
\newcommand{\coordshort}{\participantshort{\coordnr}}
\newcommand{\coordshorter}{\participantshorter{\coordnr}}

% Evaluation Partner
\newcommand{\evalnr}{\SAnr}
\newcommand{\eval}{\participant{\evalnr}}
\newcommand{\evalshort}{\participantshort{\evalnr}}
\newcommand{\evalshorter}{\participantshorter{\evalnr}}

% Dissemination Partner
\newcommand{\dissemnr}{\SAnr}
\newcommand{\dissem}{\participant{\dissemnr}}
\newcommand{\dissemshort}{\participantshort{\dissemnr}}
\newcommand{\dissemshorter}{\participantshorter{\dissemnr}}


% Personalised comments for each author
\newcommand{\jbcomment}[1]{\comment{JB}{#1}} % by Juliana Bowles, USTAN
\newcommand{\vjcomment}[1]{\comment{VJ}{#1}} % by Vladimir Janjic, UOD
\newcommand{\cbcomment}[1]{\comment{CB}{#1}} % by Vladimir Janjic, UOD
\newcommand{\accomment}[1]{\comment{AC}{#1}} % by Andy Comment, UOD

%% Related Projects
\newcommand{\advanceproject}{\mbox{\textsc{Advance}}\xspace}
\newcommand{\embounded}{\mbox{\textsc{EmBounded}}\xspace}
\newcommand{\paraphrase}{\mbox{\textsc{ParaPhrase}}\xspace}
\newcommand{\rephrase}{\mbox{\textsc{RePhrase}}\xspace}
\newcommand{\science}{\mbox{\textsc{SCIence}}\xspace}
\newcommand{\hipeac}{\mbox{\textsc{HiPEAC}}\xspace}
\newcommand{\tacle}{\mbox{\textsc{Tacle}}\xspace}
\newcommand{\discovery}{\mbox{\textsc{Discovery}}\xspace}
\newcommand{\paraphrasing}{\mbox{\textsc{ParaPhrasing}}\xspace}
\newcommand{\paraformance}{\mbox{\textsc{ParaFormance}}\xspace}
\newcommand{\teamplay}{\mbox{\textsc{TeamPlay}}\xspace}
\newcommand{\TeamPlay}{\teamplay}
\newcommand{\grow}{\mbox{\textsc{GROW}}\xspace}
\newcommand{\weobserve}{\mbox{\textsc{Weobserve}}\xspace}
\newcommand{\satie}{\mbox{\textsc{SATIE}}\xspace}
\newcommand{\sinapse}{\mbox{\textsc{SINAPSE}}\xspace}
\newcommand{\blogdas}{\mbox{\textsc{BLogDAS}}\xspace}
\newcommand{\primal}{\mbox{\textsc{PRIMAL}}\xspace}
\newcommand{\sthreeai}{\mbox{\textsc{S3AI}}\xspace}
\newcommand{\serums}{\mbox{\textsc{SERUMS}}\xspace}



\begin{document}
\pagenumbering{arabic} % for pageslts

\begin{titlepage}

\begin{center}
{\Huge \textsc{\TheProject}}
\end{center}

\begin{tabular}{lp{5in}r} %something strange about the spacing...
\textbf{Title of Proposal:}&\hspace*{-7cm}\textbf{Intelligent Security and Privacy for AI-Based Big Data Analytics } & \\[4ex] 
\textbf{Date of preparation:} &\hspace*{-3cm} \textbf{\today} & \comment{}{$
$Revision: 0.0$ $}\\[4ex]
\textbf{List of participants} & & \\[-1ex]

{{\textcolor{white}{https://www.overleaf.com/project/5e5e45121e493b000149fe20}}}
\end{tabular}

%% Participants Table
\newcounter{p}
\begin{center}
\begin{tabular}{|l|p{5in}|l|l|}\hline
\textbf{Participant no} & \textbf{Participant organisation name} & \textbf{Country}\\ \hline 
1 (Coordinator) & {\sc \longparticipant{1}} \hfill (\shortparticipant{1}) & \country{1}  \\ \hline
\forloop{p}{2}{\value{p} < \theparticipant}{%
\thep & {\sc \longparticipant{\thep}} \hfill  (\shortparticipant{\thep}) & \country{\thep}  \\ \hline}%
\theparticipant & {\sc \longparticipant{\theparticipant}} \hfill  (\shortparticipant{\theparticipant})& \country{\theparticipant}  \\ \hline
\end{tabular}\end{center}
%https://www.overleaf.com/project/5e5e45121e493b000149fe20

\tableofcontents

\end{titlepage}

% \input{snags}
\newpage


%\pagenumbering{roman}

% ---------------------------------------------------------------------------    
%  Section 1: Excellence
% ---------------------------------------------------------------------------

%\pagebreak

Contributions by partners:

\subsection{IBM}
\subsubsection{ML-based Synthetic Data Fabrication}
 
We propose a Generative Adversarial Network (GAN) based method and an algorithm for structured synthetic data fabrication. Generative models are one of the most trending and promising approaches towards the goal of fabrication realistic data. i.e. synthetic data that is ideally indistinguishable from the real data. Such generative models show high potential in their ability to learn the natural features of a data set (unlike applying various predefined data analyses). Generating realistic data using such an approach has been already shown very good results in several domains and seems to be very promising in others. While generative ML models, and specifically GANs, are an active and trendy research topic in academia, with several tools available out there on the market, most of the works are devoted to unstructured data i.e. a synthetic image, video, text, and sound. Our proposal focuses on the fabrication of realistic high-quality structural data which is especially challenging in presence of human-defined constraints. The ability to add human-defined constants to a "trained" GAN is another important differentiator of this research work..
 
GAN-based data fabrication approach provides a significant improvement over the rule-based data fabrication tool of IBM. It enables to avoid a manual effort of rules definition / modeling. Rule definition is a laborious, time and resource-consuming process, involving data analysis as a prerequisite. Dealing with real-world data might be even more challenging in presence of data irregularities and anomalies, or in cases when intrinsic data dependencies and constraints are difficult to comprehend.

\subsubsection{AI-based Breach and Attack Simulation Platform}

Security controls are powerful tools for any organization, but they can be complex and difficult to manage. An enterprise anti-malware platform may have dozens of pages of settings and configuration options. Setting something incorrectly can have consequences ranging from leaving the company open to attack through preventing users from getting their jobs done.
Automatic Breach and Attack Simulation (ABAS) is the answer to the question of how to make sure these weaknesses are found and addressed without breaking the network or the bank. At its core, ABAS is a platform designed to perform actions that closely mimic real threat actions to determine if they are caught by the organization security controls.
ABAS uses a set of complex attack scenarios that attempt to bypass security control systems to reach a specific goal. If that goal can be reached (such as traffic making it through a firewall or an email being delivered to an end recipient), then the ABAS platform has helped to uncover a flaw in that control that needs to be remediated. An ABAS platform can simulate phishing attacks, malware attacks on endpoints, data exfiltration and sophisticated advanced persistent threats employing lateral movement.

We intend to explore and develop a prototype of an automated extendable Breach and Attach Simulation technology. Our goal is to avoid as much as possible a manual effort of a new security attack definition. We plan to provide a platform that enables to define security attack scenarios in abstract way as templates and then to let the tool automatically create multiple instances of each scenario. Moreover, we plan to apply advanced ML-based techniques to automate the process of discovering new attacks and extracting properties of the target computer system.

\subsection{Yagaan}
\begin{itemize}

\item Detection of vulnerabilities in the source code of software, based on SAST and Machine Learning for (a selection of) programming languages (to be defined) => quite similar to our contribution to the Elysian project

\item Detection of vulnerabilities in the source code of software, based on Deep Learning => We start exploring this area on a SAP Labs / BPI France grant. But there is a prerequisite to keep in mind : deep Learning needs huge amount of appropriate data for training that YAG is not in a position to generate : it should be an input to the project

\item Based on the source code of an application, extraction of potential counter measures which have not been implemented (remediation recommendations) and that could feed an attack simulation platform or a self healing process.

- (low TRL...) qualification and validation of machine learning training cases driven by a community of code reviewers. The goal is to capture the training cases that different users of a community, with uncertain skills or trying to alter the vulnerability detection tool, produce when providing additional training to the YAG-Suite machine learning + automate the process of selecting relevant training cases against non relevant (or hostile) ones.

\end{itemize}

\subsection{St Andrews}

\subsubsection{Refactoring Tool Support}
Using refactoring to transform programs into more secure equivalents. USTAN have been developing the Paraformance tool (http://www.paraformance.com/) that refactors programs into versions that use parallelism. This could be extended to deal with security aspects. I have already done some work in this area using the idea of a semi-interleaved ladder, where we refactor C programs with a conditional branch over a secret variable, into one that is semi-interleaved, with equal branching profilings. We do this by having a system of sound rewrites underneath, which search for a rewriting s.t. it conforms to the security pattern.  The searching of the rewrites here could be performed using machine learning and AI techniques. In fact, for the semi-interleaved ladder, we use a trivial Taboo search, but this could be replaced with more intelligent search mechanisms, depending on the complexity of the search space.  This is just one example, and there are likely many such refactorings and patterns we could exploit here. I do wonder though if we need something in the project that deals with compiler/runtime systems, as security is often parameterized by its non-functional footprint. A compiler/runtime system will often try to optimize that footprint, resulting in less secure programs. Any transformation that we apply at the source level may not apply at the execution level due to the optimisations. Perhaps this is something we need to think about?

\subsubsection{Contract Specification Language}
In Teamplay (https://www.teamplay-h2020.eu/), USTAN developed the Contract Specification Language (CSL), a DSL for providing the developer with a way to capture non-functional requirements of their software (such as timing, power and security) and also to express verifiable assertions in the code, as provable contracts. These contracts are proved using an underlying system of dependent types (using a language called Idris), which give proofs that the assertions in the software hold. So far, we have mostly focused on timing and energy, but this could be extended to deal with security aspects. One such example could be a side-channel attack, where a vulnerable key is exposed in a conditional. Timing/power profiling of the execution of the code can lead to an attacker establishing the values of the vulnerable keys, depending on which branch is executed. A way around this attack would be to ensure both branches exhibit the same timing/power profiles for all executions, stopping the branch information from being leaked to a hacker. This could be expressed using CSL and/or proved using our system.

\subsubsection{Formal Verification}

\subsection{SCCH}
\begin{itemize}
\item \emph{Monitoring strategies for Security}: We propose to use runtime monitoring of security properties as system reconfiguration triggers. The plan is to use reconfiguration as an attack prevention or intrusion tolerance mechanism, which in both cases will be proven to preserve security invariants on the system. Certain classes of attacks as blended or low level are by nature challenging to model formally. Indeed, although being able to model intrusion consequences, we very often struggle to model accurately enough environments and attackers in order to capture attack execution and exhibit insecure states. This is where monitoring greatly helps. We can use it to improve the security of existing systems by adding monitoring layers to detect complex attack classes or intrusions (e.g. using IA monitoring) and feed the system back with events to execute accordingly. We can monitor different kind of threats according to the selected case studies. For instance, we could monitor denial of service or detect and prevent unauthorized participants.

\item \emph{Static and run-time adaptation}: Using our ground formal model for reflective systems we could define two specific adaptation operators and its associated well definedness proof obligation guaranteeing their applicability. The first operator considers a static adaptation, which corresponds to the case where a system is restarted after an adaptation situation is identified. The operator must return the adapting system to a safe initial state. The second operator considers the case of runtime adaptation, in which control is transferred to the adapting system, in a state related to the state of the adapted system using an adaptation relation. This case could lead to different control transfer strategies being investigated in the project (transfer state equivalence, degrade, or upgrade modes). In both cases, the adapting system is selected using selection operations. Note that systems can be seen as a labelled transition system and each adaptation action can be seen as an operation establishing an adaptation relationship between the source  system  (adapted  system) and  the  target  system  (adapting  system).  This  relationship  must  fulfil  the  proof obligation resulting from the adaptation operation. At this level, a collection of systems and their adaptation relationships are obtained. The objective is to build a library of labelled transitions systems linked by a correct adaptation relation, where correctness is established by discharging the proof obligations of depending on the type of adaptation.  The relation linking such systems is associated to meta-data (e.g.  adaptation mode, self-adaptation,  QoS properties, etc.)  to offer more exploitation capabilities of the defined library.

\item \emph{Self Adaptation/Self* systems/autonomous}: Here the idea is to rely on reflexive modelling to address the case of self-adaptation (healing), which is a critical operation for ensuring autonomous system behavior. It formalises an operator allowing a running system to change its  current running state after (a) identifying adaptation situation given by a monitoring action and (b) using a selection of self-adaptation substitutions.


\item Design and Evaluation of Trustworthy AI Systems
Trustworthy AI refers to the development, deployment, and use of AI by consumers, organizations, society in ways that not only ensure its compliance with all relevant laws and its robustness but especially its adherence to general ethical principles. The five principles of ethical AI (i.e. beneficence, non-maleficence, autonomy, justice, and explicability) need to be taken into account during development of machine learning and deep learning based AI systems. Despite the importance of ethical AI principles, their major limitation is concerning the fact that principles are highly general and provide little to no guidance for how they can be transferred into practice. The overall aim is to facilitate transfer of trustworthy AI principles into practice via fulfilling following aims: 
1) evaluating the privacy-leakage by an AI system; 
2) evaluating the explainability of an AI system; 
3) evaluating the transferability of an AI system from one domain/application to another domain/application; 
4) quantifying the uncertainties associated to AI models; 
5) optimizing the inherent tradeoffs between ethical AI principles such as optimization of the privacy-explainability-transferability tradeoff.
\end{itemize}




\section{Excellence}

%\begin{figure}[tp]
%  \begin{center}
%  \vspace{-5mm}
%  \includeimage[scale=0.75]{DigitalFortress-Vision-v3.png}
%%  \vspace{-3cm}
%  \caption{The \TheProject{} Vision}
%  \label{fig:vision}
%  \end{center}
%  \end{figure}





\begin{mdframed}[backgroundcolor=blue!5]
\emph{\TheProject's main goal is to XX}
\end{mdframed}





\subsection{Aims and Objectives}
\label{sect:objectives}


The specific \emph{aims} of the \TheProject{} project are:

\begin{description}
\item[Aim 1:] XX


\end{description}

The corresponding concrete \emph{objectives} are: 

%\subsubsection{Detailed Description of the Objectives}

\subsubsection*{Objective 1: XX}
% Secure Patterns for Distributed Data Processing
\vspace{-6pt}
XX

\newpage

\label{bibliography}
\addcontentsline{toc}{section}{References}

%\bibliographystyle{abbrv}
%\bibliography{bibliography_ustan}
%\bibliography{bibliography_scch}

\end{document}
